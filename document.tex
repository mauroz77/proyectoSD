%% LyX 2.1.4 created this file.  For more info, see http://www.lyx.org/.
%% Do not edit unless you really know what you are doing.
\documentclass[twocolumn,journal]{mwrep}
\usepackage[T1]{fontenc}
\setcounter{secnumdepth}{3}
\setcounter{tocdepth}{3}
\usepackage{array}
\usepackage[unicode=true,
 bookmarks=true,bookmarksnumbered=true,bookmarksopen=true,bookmarksopenlevel=1,
 breaklinks=false,pdfborder={0 0 0},backref=false,colorlinks=false]
 {hyperref}
\hypersetup{pdftitle={Your Title},
 pdfauthor={Your Name},
 pdfpagelayout=OneColumn, pdfnewwindow=true, pdfstartview=XYZ, plainpages=false}

\makeatletter

%%%%%%%%%%%%%%%%%%%%%%%%%%%%%% LyX specific LaTeX commands.
%% Because html converters don't know tabularnewline
\providecommand{\tabularnewline}{\\}

%%%%%%%%%%%%%%%%%%%%%%%%%%%%%% User specified LaTeX commands.
% for subfigures/subtables
\usepackage[caption=false,font=footnotesize]{subfig}

\makeatother

\begin{document}





\title{Definición del Tema de Investigación.}


\author{Wilson Esteban Vélez Morales\thanks{Wilson Esteban Velez Morales estudiante de Maestría en Ingenieria
con Énfasis en Sistemas y Computación, Escuela de Ingeniería de Sistemas
y Computación, Universidad del Valle, Cali, Colombia, e-mail: \protect\href{http://wilson.velez@correounivalle.edu.co}{wilson.velez@correounivalle.edu.co}.}}

\maketitle
Wilson Esteban Velez Morales : Definición del Tema de Investigación.
\begin{abstract}
En éste documento se busca definir el tema de investigación para el
desarrollo del trabajo de grado de la Maestría en Ingeniería con Énfasis
en Sistemas y Computación de la Universidad del Valle, tomando como
base de estudio el concepto de Gobernanza de Desarrollo de Software
(SDG por sus siglas en inglés Software Development Governance).
\end{abstract}
Software Development Governance, Software Governance, IT Governance,
Management, Software Processes, Decision Making, Roles and Responsabilities,
Team Management, Knowledge Management and Team Knowledge-Based, Software
Architecture.




\section{Autores destacados}
\begin{itemize}
\item Yael Dubinsky, IBM Research \textendash{} Haifa Software and system
engineering Dirección de correo verificada de il.ibm.com
\item Orit Hazzan, Technion \textendash{} Israel Institute of Technology
Computer science education, Teaching human aspects of software engineering,
Agile software development, Collegiate mathematics education Dirección
de correo verificada de techunix.technion.ac.il 
\item Paul Bannerman Afiliación desconocida Dirección de correo verificada
de nicta.com.au
\item Clay Williams, Cohere Health Computer Science Dirección de correo
verificada de coherehealth.com
\item Maria-Eugenia, Iacob Associate professor, University of Twente enterprise
architecture, business process management, service oriented architecture,
business rules, model-driven design Dirección de correo verificada
de utwente.nl 
\item Pascal van Eck, University of Twente Enterprise Architecture, IT Governance
Dirección de correo verificada de utwente.nl 
\end{itemize}

\section{Revistas y confere\index{conferencias}ncias}
\begin{itemize}
\item International Conference on Software Engineering (ICSE)
\item ACM Association for Computing Machinery
\item IEEE International Conference on Management of Innovation and Technology
(ICMIT)
\item IEEE International Conference on Global Software Engineering
\item IEEE CONFERENCE PUBLICATIONS
\item Americas Conference on Information Systems (AMCIS)
\item European Conference Software Architecture, ECSA 
\end{itemize}

\section{Grupos de investigación}
\begin{itemize}
\item IBM Research
\item The Center for Information Systems Research (CISR) conducts field-based
research on issues related to the management and use of information
technology (IT) in complex organizations.
\item IT Governance Institute
\item ISACA Information Systems Audit and Control Association
\item IRMA Information Resources Management Association
\item SEI Software Engineering Institute
\end{itemize}

\section{Resumen.}

a gobernanza de desarrollo de software o SDG (por sus siglas en
inglés Software Development Governance) se conforma de:

Un componente estático que establece las cadenas de responsabilidad,
autoridad y comunicación con el fin de empoderar a las personas dentro
de la organización.

Un componente dinámico que establece las medidas y los mecanismos
de control que permitan habilitar a los desarrolladores de software,
gerentes de proyecto, y otras personas dentro de la organizacion,
llevar a cabo sus roles y responsabilidades. \cite{Chulani2008}

El SDG tiene como fin lograr la alineación del desarrollo de software
con los objetivos estratégicos del negocio, proporcionando a los equipos
de trabajo y en general a las organizaciones, las habilidades para
dirigir de forma efectiva el negocio de desarrollo de software \cite{dubinsky_2nd_2009},
convirtiéndose en un punto clave entre los procesos de desarrollo
de software y el negocio.

Los equipos de desarrollo requieren proporcionar mayor valor de manera
efectiva y eficiente, controlando los riesgos operacionales y cumpliendo
con las regulaciones; el SDG permite empoderalos para alcanzar las
metas de los proyectos. Un componente importante que apoya la implementación
del SDG es la arquitectura de software, logrando traducir las necesidades
de la organización a los equipos de trabajo, donde se establece el
impacto técnico y herramientas de implementación. \cite{Dubinsky2010}
Para obtener los resultados esperados en el empoderamiento y la comunicación
entre los equipos de desarrollo, se debe contar con una adecuada transferencia
de conocimiento entre los integrantes, generando información de métodos
y herramientas técnicas que permitan alcanzar los objetivos establecidos.
\cite{Manteli2011}


\section{Aporte al tema enfocado.}

Generar un modelo basado en la gobernanza de desarrollo de software
que pueda ser aplicado en las organizaciones, con el fin de solucionar
inconvenientes como la toma de decisiones en los equipos de desarrollo
de software, el seguimiento y control de las decisiones que se tomen,
priorización de tareas, definición de roles y responsabilidades, implementación
de metodologías y herramientas que faciliten la transferencia de conocimiento
técnico y de negocio al interior de los equipos de trabajo, por otro
lado definir un marco de trabajo que permita estandarizar los procedimientos
para la construcción de software por medio de una arquitectura de
referencia aplicable tanto al software generado en la organización,
como al software de terceros. Así mismo, proporcionar herramientas
que faciliten la toma de decisiones referentes al software, como es
la implementación de nuevas tecnologías (por ejemplo BPM) para problemas
específicos.


\section{Idea inicial.}

Generar un modelo basado en gobernanza de desarrollo de software para
su implementación dentro de la organización Bancoomeva S.A. proporcionando
herramientas de arquitectura de software que permitan estandarizar
los componentes de software realizados desde el equipo de la Gerencia
de Sistemas, así como los desarrollos que se requieran tercerizar
para las plataformas del core de negocio Iseries AS400 de IBM o las
plataformas de gestión de procesos de negocio BPM, así mismo para
los desarollos que se realicen para los canales virtuales de atención
al clientes como son Banca Móvil, Oficinas Virtuales, Audio IVR; estableciendo
los flujos de responsabilidad, autoridad y comunicación donde se definan
los canales, métodos y herramientas para transferencia de conocimiento
técnico y de negocio; así mismo determinar las políticas, controles
y mecanismos de medición que contribuyan a la mejoría continua del
modelo de gobernanza obteniendo resultando en los niveles de madurez
del mismo. 


\section{Bibliografía anotada.}

\begin{tabular}{|c||c||>{\centering}p{35cc}|}
\hline 
Referencia & Justificación & Notas\tabularnewline
\hline 
\hline 
Software development governance and its concerns & Este articulo busca definir el SDG, aclarar dudas con el concepto
 y la aplicacion del mismo. & Este documento busca clarificar la relacion del SDG con la administración
y procesos. En el documento plantean que el objetivo del SDG es asegurar
que los procesos de negocio de una organizacion de software reuna
los requerimientos estratégicos de la organización.\tabularnewline
\hline 
\hline 
Workshop on \{Software\} \{Development\} \{Governance\} (\{SDG\} 2008) & Primer taller sobre Gobernanza de Desarrollo de Software en el marco
de la 30 Conferencia Internacional de Ingenieria de Software ICSE
2008 , donde se empieza a abordar el tema para su definición y alcance. & El talle busca El deseo de este taller entender el valor que los proyectos
de desarrollo de software provee, asi como los riesgos que conlleva.
El campo que direcciona estas preocupaciones es conocido como Gobernanza
de Desarrollo de Software.\tabularnewline
\hline 
\hline 
2nd Workshop on software development governance (SDG) & Segundo taller sobre SDG en en el marco de la 30 Conferencia Internacional
de Ingenieria de Software ICSE 2009 , se continua con el taller propuesto
y realizado por primera vez en 2008 & En el documento se indica que La funcion principal del SDG es lograr
una alineación estrategica con el negocio. la exploración del SDG
como un paso importante en el ingenieria de software. La implementación
de gobernanza a traves de herramientas y tecnicas que proveen a los
equipos y organizaciones la habilidad para dirigir efectivamente el
negocio del desarrollo de software.\tabularnewline
\hline 
\hline 
Software Development Governance (SDG) Workshop & Tercer taller de SDG en el marco de la 30 Conferencia Internacional
de Ingenieria de Software ICSE 2010  & Se combinaron dos talleres SDG y LMSA (liderazgo y gestión en arquitectura
de software) , plantea que SDG es el punto clave entre el proceso
de desarrollo de software, el negocio y el contexto regulatorio en
el cual el proceso opera, enmarca el proceso SDG en un modelo iterativo
que depende de las preocupaciones del negocio y la organización.\tabularnewline
\hline 
\hline 
 &  & \tabularnewline
\hline 
\end{tabular}

\bibliographystyle{IEEEtran}
\nocite{*}
\bibliography{\string"E:/Maestria Ingenieria/Semestre I/Introducción a la Investigación/Articulos/definicion del tema B\string"}

\end{document}
